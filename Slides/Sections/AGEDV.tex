%%%%%%%%%%%%%%%%%%%%%%%%%%%%%%%%%%%%%%%%%%%%%%%%%%%%%%%%%%%%%%%%%%%%
% Authors: A. Herrera-Poyatos, F. Herrera
% Tittle: Algoritmo memético equilibrado con diversificación voraz
% 							 CAEPIA 2015
%%%%%%%%%%%%%%%%%%%%%%%%%%%%%%%%%%%%%%%%%%%%%%%%%%%%%%%%%%%%%%%%%%%%

\section[AGEDV]{Algoritmo Genético Equilibrado con Diversificación Voraz}

	\subsection*{Diversificación voraz}

		\begin{frame}{Diversificación voraz: hibridación con algoritmos voraces aleatorizados}

			\fontsize{8}{10}\selectfont

			\begin{tcolorbox}[colback=blue!5,colframe=blue!30]
				\centering
				\textbf{Idea:} Sustituir los cromosomas de la población que sean similares a otros por nuevos cromosomas voraces.	
			\end{tcolorbox}

			\begin{tikzpicture}[
				every node/.style={ellipse,inner sep=4pt, text width=2cm, align=center, node distance=2cm}]
				
				\node [fill=blue!20] (i) {Inicializar \\ $P$};
				\node [fill=blue!40, right = of $(i)$] (e) {Evolucionar \\ $P$};
				\node [fill=blue!60, right= of $(e)$] (dv) {Diversificación voraz};
				
				\path[every node/.style={font=\sffamily\small}]
				(i) edge [->, >=stealth', line width=0.5mm] node [right] {} (e)
				(e) edge [->, >=stealth', bend right, line width=0.5mm] node [right] {} (dv)
				(dv) edge [->, >=stealth', bend right, line width=0.5mm] node [left] {} (e);
			\end{tikzpicture}

			
			\textbf{Algoritmos voraces aleatorizados:}

			\begin{itemize}
				\item Producen cromosomas diversos y de calidad.
				\item Sinergia con la operación de cruce
			\end{itemize}


			\begin{tcolorbox}[colback=blue!5,colframe=blue!30]
				\centering
				\textbf{Soluciona el problema de la diversidad}	
			\end{tcolorbox}

		\end{frame}

	\subsection*{Otras componentes}

		\begin{frame}{Componentes de AGEDV}
	
			\fontsize{8}{10}\selectfont
			\kern -0.8cm
			\begin{tikzpicture}[
				every node/.style={circle, inner sep=1pt, text width=1.8cm, align=center, node distance=2cm}]
				
				\node [fill=blue!15] (i) {Inicializar $P$};
				\node [fill=blue!25, right = of $(i)$] (s) {Selección};
				\node [fill=blue!55, right = 5cm of $(s)$] (co) {Competición};
				\node [fill=blue!40, above = 0.3cm of {$(s)!0.5!(co)$}] (c) {Cruce};
				\node [fill=blue!70, below = 0.3cm of {$(s)!0.5!(co)$}] (dv) {Diversificación voraz};
				
				\path[every node/.style={font=\sffamily\small}]
				(i) edge [->, >=stealth', line width=0.5mm] node {} (s)
				(s) edge [->, >=stealth', line width=0.5mm] node {} (c)
				(c) edge [->, >=stealth', line width=0.5mm] node {} (co);
		
				\path[every node/.style={font=\sffamily\small}]
				(co) edge [->, >=stealth', line width=0.5mm] node {} (dv);
		
				\path[every node/.style={font=\sffamily\small}]
				(dv) edge [->, >=stealth', line width=0.5mm] node {} (s);
			
			\end{tikzpicture}

			\begin{itemize}
				\item\textbf{Selección aleatoria adyacente:} 			\fontsize{6}{10}\selectfont \twopartdef{Se ordena aleatoriamente la población.} {Se selecciona las parejas de cromosomas adyacentes.}
				\fontsize{8}{10}\selectfont
				\item Cada pareja genera un hijo.
				\item \textbf{Competición entre padres e hijos.} Cada hijo compite con su padre directo.
			\end{itemize}

			\begin{tcolorbox}[colback=blue!5,colframe=blue!30]
				\centering
				\fontsize{10}{10}\selectfont
				\textbf{Buscamos la sinergia entre las componentes.}	
			\end{tcolorbox}

		\end{frame}