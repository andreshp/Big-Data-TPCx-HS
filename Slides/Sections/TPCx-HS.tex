%%%%%%%%%%%%%%%%%%%%%%%%%%%%%%%%%%%%%%%%%%%%%%%%%%%%%%%%%%%%%%%%%%%%
% Authors: A. Herrera-Poyatos, F. Herrera
% Tittle: Algoritmo memético equilibrado con diversificación voraz
% 							 CAEPIA 2015
%%%%%%%%%%%%%%%%%%%%%%%%%%%%%%%%%%%%%%%%%%%%%%%%%%%%%%%%%%%%%%%%%%%%

\section{TPCx-HS}

	\subsection*{¿Por qué usar TPCx-HS?}

		\begin{frame}{Información sobre TPCx-HS}
			\begin{figure}[H]
				\centering
				\includegraphics[width=5cm]{./Images/tpc-logo.png}
			\end{figure}
			
			{\color{ChetwodeBlue}\large Carga de trabajo de TPCx-HS}
			
			\begin{itemize}
				\item HSGen: genera los datos según un factor de escala.
				\item HSDataCheck: comprueba el cumplimiento del conjunto de datos.
				\item HSSort: ordena los datos según un orden total.
				\item HSValidate: comprueba si los datos obtenidos son válidos.
			\end{itemize}		
		\end{frame}
	
	\subsection*{Funcionamiento de TPCx-HS}	

		\begin{frame}{Ejecución de TPCx-HS}
				\begin{columns}[c]
					\column{.6\textwidth}
					
					\begin{itemize}
						\item Fase 1: Generación de los datos.
						\item Fase 2: Verificación de la validez de los datos.
						\item Fase 3: Ordena de los datos.
						\item Fase 4: Verificación de la viabilidad de los datos.
						\item Fase 5: Validación de los datos.
					\end{itemize}
					
					\column{.7\textwidth}
					\begin{figure}[H]
						\centering
						\includegraphics[width=5cm]{./Images/executionsTPC.png}
					\end{figure}

				\end{columns}
	
		\end{frame}
		
	\subsection*{Medida del rendimiento}	
			
		\begin{frame}{Rendimiento}		
			{\color{ChetwodeBlue}\large Medida del rendimiento.}
								
					$$ HSph@SF = \frac{SF}{T/3600} $$				
				
			{\color{ChetwodeBlue}\large Medida del rendimiento-precio.}
			
					$$ \$/HSph@SF = \frac{P}{HSph@SF} $$
					
			\begin{tcolorbox}[colback=blue!5,colframe=blue!15]
				\textbf{Parámetros:}
		
				\begin{itemize}
					\fontsize{10}{10}\selectfont
					\item SF: factor de escala escogido.
					\item T: tiempo total de las dos ejecuciones.
					\item P: costo del sistema bajo estudio.
				\end{itemize}
			\end{tcolorbox}
		\end{frame}
				 